% ----------------------------------------------------------
\chapter{Introdução}\label{cap:introducao}
% ----------------------------------------------------------

O avanço contínuo das tecnologias móveis, aliado à popularização da computação em nuvem, transformou radicalmente a forma como as pessoas interagem, consomem e organizam suas rotinas cotidianas. Dispositivos portáteis tornaram-se ferramentas indispensáveis para gestão pessoal e familiar, incorporando desde o controle financeiro até o planejamento de atividades domésticas. Nesse contexto, a digitalização das tarefas rotineiras representa não apenas conveniência, mas também um caminho para otimização de tempo, redução de desperdícios e aumento da eficiência no uso de recursos.

Entre essas tarefas, o gerenciamento de listas de compras assume papel de destaque. A necessidade de planejar, registrar e compartilhar informações sobre produtos e orçamentos é uma atividade recorrente em lares modernos, especialmente em um cenário de crescente valorização do consumo consciente e da colaboração familiar. Contudo, apesar da disponibilidade de diversos aplicativos de listas no mercado, muitos carecem de integração em tempo real, colaboração multiusuário e análise inteligente de gastos, o que limita sua aplicabilidade prática e o potencial de apoio à decisão.

De acordo com \textcite{armbrust2010}, a computação em nuvem redefiniu o modo como aplicações são projetadas e escaladas, permitindo que soluções distribuídas e colaborativas sejam implementadas com baixo custo operacional e alta disponibilidade. Essa evolução tecnológica abre espaço para sistemas capazes de operar em múltiplas plataformas, com persistência de dados e sincronização em tempo real, elementos essenciais para o desenvolvimento de aplicativos modernos e colaborativos. Nesse cenário, o \textit{Firebase} se destaca como uma plataforma de serviços integrados que oferece autenticação, armazenamento e mensageria, enquanto o \textit{Flutter} proporciona uma base única de código para múltiplos dispositivos, mantendo consistência visual e desempenho nativo.

O aplicativo proposto neste trabalho surge, portanto, da necessidade de unir essas duas vertentes tecnológicas — \textit{Flutter} e \textit{Firebase} — em uma solução completa e acessível para gerenciamento de listas de compras. O objetivo é fornecer um ambiente colaborativo em que vários usuários possam criar, editar e compartilhar listas em tempo real, além de acompanhar despesas e padrões de consumo por meio de relatórios dinâmicos. A integração de ferramentas de análise visual e o uso de sincronização em nuvem permitem transformar uma atividade cotidiana em um processo estruturado e inteligente.

A relevância deste projeto está diretamente associada à crescente demanda por sistemas que combinem praticidade e inteligência de dados, especialmente em um contexto de conectividade constante e múltiplos dispositivos. O uso de tecnologias modernas possibilita não apenas uma experiência fluida e multiplataforma, mas também a adoção de práticas mais sustentáveis e colaborativas na rotina doméstica. Conforme destaca \textcite{sommerville2011}, a engenharia de software moderna deve priorizar a confiabilidade, a manutenibilidade e a usabilidade das aplicações, princípios que norteiam o desenvolvimento deste trabalho.

Do ponto de vista metodológico, a implementação do sistema foi conduzida segundo uma abordagem ágil, com ciclos iterativos de concepção, prototipação, testes e validação contínua. Essa estratégia permitiu ajustes progressivos conforme as demandas do projeto e a avaliação de desempenho em cenários reais. Além disso, o uso de versionamento de código e controle colaborativo via \textit{GitHub} garantiu rastreabilidade e consistência entre as etapas de desenvolvimento, favorecendo a integração incremental dos módulos do sistema.

Os resultados obtidos demonstram que a aplicação atende aos objetivos estabelecidos, apresentando desempenho satisfatório, usabilidade adequada e robustez funcional em ambiente colaborativo. A arquitetura modular e reativa adotada, aliada à infraestrutura em nuvem, viabiliza futuras extensões, como integração com APIs de supermercados, implementação de sincronização \textit{offline-first} e expansão para plataformas web e desktop. 

Em termos de contribuição prática, o sistema se posiciona como uma ferramenta de apoio à gestão doméstica e à educação financeira familiar, oferecendo uma interface simples e eficiente para o controle de gastos e a tomada de decisão. No campo acadêmico, o trabalho reforça o potencial do desenvolvimento multiplataforma com o uso de tecnologias emergentes como o \textit{Flutter} e o \textit{Firebase}, alinhando inovação tecnológica a um problema cotidiano de relevância social.

Este trabalho está organizado da seguinte forma: o Capítulo~\ref{cap:objetivos} apresenta os objetivos; o Capítulo~\ref{cap:fundamentacao} traz a fundamentação teórica, destacando os conceitos de computação em nuvem, desenvolvimento multiplataforma e colaboração digital; o Capítulo~\ref{cap:metodologia} descreve a metodologia adotada; o Capítulo~\ref{cap:desenvolvimento} detalha o processo de implementação; o Capítulo~\ref{cap:resultados} apresenta os resultados obtidos e as discussões; e, por fim, o Capítulo~\ref{cap:conclusao} sintetiza as conclusões e contribuições do trabalho.

Conforme reforça \textcite{pressman2016}, o desenvolvimento de sistemas modernos requer integração contínua entre projeto, implementação e validação — princípio que fundamentou cada etapa deste trabalho, desde a concepção até os testes finais.
\vspace{12pt}