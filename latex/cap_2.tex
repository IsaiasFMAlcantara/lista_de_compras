% ----------------------------------------------------------
\chapter{Objetivos}\label{cap:objetivos}
% ----------------------------------------------------------

O desenvolvimento de soluções digitais voltadas à organização doméstica e ao consumo consciente exige não apenas recursos tecnológicos, mas também uma estrutura que favoreça a colaboração e o controle inteligente de informações. Dessa forma, este capítulo define o objetivo central e os objetivos específicos do presente trabalho, alinhando o escopo técnico à relevância prática e acadêmica da pesquisa.

\section{Objetivo Geral}

Desenvolver um aplicativo multiplataforma para gerenciamento inteligente de listas de compras, utilizando o \textit{framework Flutter} e os serviços do \textit{Firebase}, de forma a permitir controle dinâmico de itens, organização por categorias e sincronização em tempo real dos dados do usuário. O sistema visa otimizar o processo de planejamento de compras, reduzir desperdícios e facilitar a gestão individual das informações de consumo.

\section{Objetivos Específicos}

Para atingir o objetivo geral proposto, o trabalho busca cumprir as seguintes metas técnicas e funcionais:

\begin{itemize}
    \item \textbf{Projetar e implementar uma arquitetura modular e escalável}, estruturada em camadas (apresentação, controle e modelo), aplicando boas práticas de engenharia de software e promovendo manutenção facilitada e baixo acoplamento entre módulos;
    
    \item \textbf{Implementar autenticação e gerenciamento de usuários} utilizando o serviço \textit{Firebase Authentication}, assegurando privacidade e integridade dos dados por meio de mecanismos de autenticação baseados em credenciais seguras;

    \item \textbf{Modelar e estruturar os dados no \textit{Cloud Firestore}} para representar listas, itens e categorias, garantindo consistência, escalabilidade e atualização em tempo real;
    
    \item \textbf{Assegurar sincronização imediata dos dados do usuário} por meio de \textit{listeners} do Firestore, permitindo que modificações realizadas nas listas sejam refletidas rapidamente na interface;

    \item \textbf{Construir interface e fluxo de navegação} aplicando o \textit{framework} GetX para gerenciamento de estado, controle de rotas e reatividade entre as camadas de apresentação e lógica de negócio;

    \item \textbf{Documentar o processo de desenvolvimento e as decisões técnicas adotadas}, conforme as normas acadêmicas e metodológicas, garantindo reprodutibilidade e clareza no processo de construção do sistema;

    \item \textbf{Propor diretrizes para evoluções futuras}, incluindo notificações push, sincronização \textit{offline-first} e integração de novos recursos que ampliem a experiência do usuário.
\end{itemize}

O conjunto desses objetivos visa não apenas a entrega de um produto funcional, mas também a consolidação de uma base metodológica e tecnológica capaz de servir como referência para futuras aplicações colaborativas orientadas a dados e acessíveis em múltiplas plataformas.
