% ------------------------------------------------------------------------
% Modelo de TCC do curso de Engenharia de Controle e Automação - UFSC/Campus Blumenau
% ------------------------------------------------------------------------

\documentclass[
12pt,
oneside,
a4paper,
chapter=TITLE,
section=TITLE,
english,
brazil
]{abntex2}

% ------------------------------------------------------------------------
% PACOTES BÁSICOS E CONFIGURAÇÕES
% ------------------------------------------------------------------------
\usepackage[T1]{fontenc}            % UTF-8 robusto
\usepackage[utf8]{inputenc}

\usepackage{configs/eca_ufsc_bnu}   % personalização ABNTEX2

% --- GLOSSARIES (resolve erro do sort=...) ---
\usepackage[acronym]{glossaries}
\makeglossaries

% --- LISTINGS (código-fonte) ---
\usepackage{listings}
\usepackage{listingsutf8}
\usepackage{xcolor}

% Estilo + normalização de UTF-8 dentro de listings
\lstset{
  inputencoding=utf8,
  extendedchars=true,
  literate=%
   {á}{{\'a}}1 {Á}{{\'A}}1
   {à}{{\`a}}1 {À}{{\`A}}1
   {â}{{\^a}}1 {Â}{{\^A}}1
   {ã}{{\~a}}1 {Ã}{{\~A}}1
   {é}{{\'e}}1 {É}{{\'E}}1
   {ê}{{\^e}}1 {Ê}{{\^E}}1
   {í}{{\'i}}1 {Í}{{\'I}}1
   {ó}{{\'o}}1 {Ó}{{\'O}}1
   {ô}{{\^o}}1 {Ô}{{\^O}}1
   {õ}{{\~o}}1 {Õ}{{\~O}}1
   {ú}{{\'u}}1 {Ú}{{\'U}}1
   {ç}{{\c c}}1 {Ç}{{\c C}}1
   {–}{{-}}1 {—}{{-}}1 {“}{{``}}1 {”}{{''}}1 {…}{{...}}1,
  basicstyle=\footnotesize\ttfamily,
  backgroundcolor=\color{gray!10},
  frame=single,
  breaklines=true,
  captionpos=b,
  numbers=left,
  numberstyle=\tiny\color{gray},
  xleftmargin=1em,
  xrightmargin=1em,
  showstringspaces=false,
  tabsize=2,
  columns=fullflexible,
  keywordstyle=\bfseries\color{blue!70!black},
  commentstyle=\itshape\color{gray!80!black},
  stringstyle=\color{orange!70!black}
}

% Linguagem Dart (listings não traz nativo)
\lstdefinelanguage{Dart}{
  sensitive=true,
  morekeywords={abstract,as,assert,async,await,break,case,catch,class,const,continue,
  default,deferred,do,dynamic,else,enum,export,extends,extension,external,factory,false,
  final,finally,for,Function,get,hide,if,implements,import,in,interface,is,late,library,
  mixin,new,null,on,operator,part,required,rethrow,return,set,show,static,super,switch,
  sync,this,throw,true,try,typedef,var,void,while,with,yield},
  morecomment=[l]{//},
  morecomment=[s]{/*}{*/},
  morestring=[b]",
}

% (Opcional) Python reforçado
\lstdefinelanguage{MyPython}[]{Python}{
  morekeywords={self, as, with, async, await},
}

\addbibresource{pos_textual/referencias.bib}

% ------------------------------------------------------------------------
% DADOS BÁSICOS DO TCC
% ------------------------------------------------------------------------
\autor{ISAIAS FÉLIX MACHADO DE ALCANTARA}
\titulo{DESENVOLVIMENTO DE UM APLICATIVO MULTIPLATAFORMA PARA GERENCIAMENTO INTELIGENTE DE LISTAS DE COMPRAS COM FLUTTER E FIREBASE}
\subtitulo{DA COLABORAÇÃO EM TEMPO REAL À ANÁLISE DE GASTOS: UMA SOLUÇÃO PARA OTIMIZAÇÃO DA ROTINA E DO ORÇAMENTO FAMILIAR}
\orientador{Andrey Alencar Quadros}
\dia{06}
\mes{outubro}
\ano{2025}
\local{Ariquemes}
\formacao{Analista e Desenvolvedor de Sistemas}
\bancaa{Prof. Mestre Andrey Alencar Quadros (orientador)}
\bancab{Prof. Mestre Luciano Topolniak}
\bancac{Prof. Especialista Marcos Alves Faino}

\resumotcc{O presente trabalho apresenta o desenvolvimento de um aplicativo multiplataforma para gerenciamento inteligente de listas de compras, implementado com o framework Flutter e o conjunto de serviços Firebase. A proposta busca oferecer uma solução prática e colaborativa para o controle de compras domésticas, permitindo que múltiplos usuários criem, editem e compartilhem listas em tempo real, otimizando tanto o processo de aquisição de produtos quanto o acompanhamento de gastos familiares. Para atingir esse objetivo, foi adotada uma abordagem de desenvolvimento ágil, com autenticação via Firebase Authentication, armazenamento em Cloud Firestore e gráficos dinâmicos para análise de consumo. O resultado é um sistema funcional, escalável e reativo, com sincronização em tempo real e experiência de uso consistente em Android}
\palavraschave{Flutter. Firebase. Aplicativo multiplataforma. Lista de compras. Colaboração em tempo real.}

\abstracttcc{This work presents the development of a multiplatform application for intelligent shopping list management, implemented using the Flutter framework and the Firebase service suite. The proposed solution aims to provide a practical and collaborative tool for household shopping control, allowing multiple users to create, edit, and share lists in real time while optimizing product acquisition and family expense tracking. An agile development approach was adopted, integrating Firebase Authentication for user login, Cloud Firestore for data storage, and dynamic charts for consumption analysis. The resulting system is functional, scalable, and reactive, offering real-time synchronization and a consistent user experience across Android devices.}
\keywords{Flutter. Firebase. Cross-platform application. Shopping list. Real-time collaboration.}

\agradecimentostcc{
Agradeço primeiramente a Deus, pela força e sabedoria concedidas ao longo desta jornada.\\
À minha família, pelo apoio incondicional, paciência e incentivo constantes.\\
Ao meu orientador, Prof. Andrey Alencar Quadros, pela orientação precisa, disponibilidade e pelas valiosas contribuições que tornaram este trabalho possível.
}
\dedicatoriatcc{}

\contemquadros{}
\contemsiglas{sim}
\contemsimbolos{sim}

% ------------------------------------------------------------------------
% CONFIGURAÇÕES ADICIONAIS
% ------------------------------------------------------------------------
\setitemize{topsep=0pt,itemsep=0pt,leftmargin=\parindent+\labelwidth-\labelsep}
\setenumerate{topsep=0pt,itemsep=0pt,leftmargin=\parindent+\labelwidth-\labelsep}

% ------------------------------------------------------------------------
% CONFIGURAÇÕES PDF E HYPERREF
% ------------------------------------------------------------------------
\definecolor{blue}{RGB}{41,5,195}
\hypersetup{
	pdftitle={\@title}, 
	pdfauthor={\@author},
	pdfsubject={\imprimirpreambulo},
	pdfcreator={LaTeX with abnTeX2},
	pdfkeywords={Flutter, Firebase, listas de compras, multiplataforma}, 
	colorlinks=true,
	linkcolor=black,
	citecolor=black,
	filecolor=black,
	urlcolor=blue,
	bookmarksdepth=4
}

% --- (opcional, mas útil) carrega definições de siglas/símbolos antes do doc ---
%----------------- LISTA DE ABREVIATURAS E SIGLAS--------------------------------------
\siglalista{ABNT}{Associação Brasileira de Normas Técnicas}
\siglalista{API}{Application Programming Interface}
\siglalista{CRUD}{Create, Read, Update and Delete}
\siglalista{CI/CD}{Continuous Integration / Continuous Deployment}
\siglalista{DB}{Database}
\siglalista{IDE}{Integrated Development Environment}
\siglalista{MVC}{Model-View-Controller}
\siglalista{UX}{User Experience}
\siglalista{UI}{User Interface}
\siglalista{SDK}{Software Development Kit}
\siglalista{JSON}{JavaScript Object Notation}
\siglalista{HTTP}{Hypertext Transfer Protocol}
\siglalista{HTTPS}{Hypertext Transfer Protocol Secure}
\siglalista{FCM}{Firebase Cloud Messaging}
\siglalista{RTDB}{Realtime Database}
\siglalista{NoSQL}{Not Only Structured Query Language}
\siglalista{IFRO}{Instituto Federal de Educação, Ciência e Tecnologia de Rondônia}
\siglalista{ADS}{Análise e Desenvolvimento de Sistemas}
\siglalista{UX/UI}{User Experience / User Interface}
\siglalista{TCC}{Trabalho de Conclusão de Curso}
\siglalista{HTML}{HyperText Markup Language}
\siglalista{CSS}{Cascading Style Sheets}
\siglalista{Dart}{Linguagem utilizada pelo framework Flutter}
\siglalista{Firebase}{Plataforma de serviços em nuvem para aplicações móveis e web}
\siglalista{Flutter}{Framework multiplataforma desenvolvido pela Google}

%Para usar uma dada sigla ABC ao longo do texto, use \acrfull{ABC} se quiser apresentar a sigla e sua definição.
%Se quiser apresentar apenas a sigla, use \acrshort{ABC}.






%-----------------SÍMBOLOS---------------------------------------------------------------
\simbololista{t}{\ensuremath{t}}{Tempo de sincronização em segundos}
\simbololista{ms}{\ensuremath{ms}}{Milissegundos — unidade de medida usada para desempenho}
\simbololista{MB}{\ensuremath{MB}}{Megabytes — unidade de medida usada para consumo de memória}
\simbololista{s}{\ensuremath{s}}{Segundos — unidade de medida temporal em testes de performance}

%Para usar um dado símbolo SIMB ao longo do texto, use \gls{SIMB}.

% ------------------------------------------------------------------------
% INÍCIO DO DOCUMENTO
% ------------------------------------------------------------------------
\begin{document}

\selectlanguage{brazil}
\frenchspacing 
\OnehalfSpacing

\input{pre_textual/pretextual}

\textual

% 1 - Introdução
% ----------------------------------------------------------
\chapter{Introdução}\label{cap:introducao}
% ----------------------------------------------------------

O avanço contínuo das tecnologias móveis, aliado à popularização da computação em nuvem, transformou radicalmente a forma como as pessoas interagem, consomem e organizam suas rotinas cotidianas. Dispositivos portáteis tornaram-se ferramentas indispensáveis para gestão pessoal e familiar, incorporando desde o controle financeiro até o planejamento de atividades domésticas. Nesse contexto, a digitalização das tarefas rotineiras representa não apenas conveniência, mas também um caminho para otimização de tempo, redução de desperdícios e aumento da eficiência no uso de recursos.

Entre essas tarefas, o gerenciamento de listas de compras assume papel de destaque. A necessidade de planejar, registrar e compartilhar informações sobre produtos e orçamentos é uma atividade recorrente em lares modernos, especialmente em um cenário de crescente valorização do consumo consciente e da colaboração familiar. Contudo, apesar da disponibilidade de diversos aplicativos de listas no mercado, muitos carecem de integração em tempo real, colaboração multiusuário e análise inteligente de gastos, o que limita sua aplicabilidade prática e o potencial de apoio à decisão.

De acordo com \textcite{armbrust2010}, a computação em nuvem redefiniu o modo como aplicações são projetadas e escaladas, permitindo que soluções distribuídas e colaborativas sejam implementadas com baixo custo operacional e alta disponibilidade. Essa evolução tecnológica abre espaço para sistemas capazes de operar em múltiplas plataformas, com persistência de dados e sincronização em tempo real, elementos essenciais para o desenvolvimento de aplicativos modernos e colaborativos. Nesse cenário, o \textit{Firebase} se destaca como uma plataforma de serviços integrados que oferece autenticação, armazenamento e mensageria, enquanto o \textit{Flutter} proporciona uma base única de código para múltiplos dispositivos, mantendo consistência visual e desempenho nativo.

O aplicativo proposto neste trabalho surge, portanto, da necessidade de unir essas duas vertentes tecnológicas — \textit{Flutter} e \textit{Firebase} — em uma solução completa e acessível para gerenciamento de listas de compras. O objetivo é fornecer um ambiente colaborativo em que vários usuários possam criar, editar e compartilhar listas em tempo real, além de acompanhar despesas e padrões de consumo por meio de relatórios dinâmicos. A integração de ferramentas de análise visual e o uso de sincronização em nuvem permitem transformar uma atividade cotidiana em um processo estruturado e inteligente.

A relevância deste projeto está diretamente associada à crescente demanda por sistemas que combinem praticidade e inteligência de dados, especialmente em um contexto de conectividade constante e múltiplos dispositivos. O uso de tecnologias modernas possibilita não apenas uma experiência fluida e multiplataforma, mas também a adoção de práticas mais sustentáveis e colaborativas na rotina doméstica. Conforme destaca \textcite{sommerville2011}, a engenharia de software moderna deve priorizar a confiabilidade, a manutenibilidade e a usabilidade das aplicações, princípios que norteiam o desenvolvimento deste trabalho.

Do ponto de vista metodológico, a implementação do sistema foi conduzida segundo uma abordagem ágil, com ciclos iterativos de concepção, prototipação, testes e validação contínua. Essa estratégia permitiu ajustes progressivos conforme as demandas do projeto e a avaliação de desempenho em cenários reais. Além disso, o uso de versionamento de código e controle colaborativo via \textit{GitHub} garantiu rastreabilidade e consistência entre as etapas de desenvolvimento, favorecendo a integração incremental dos módulos do sistema.

Os resultados obtidos demonstram que a aplicação atende aos objetivos estabelecidos, apresentando desempenho satisfatório, usabilidade adequada e robustez funcional em ambiente colaborativo. A arquitetura modular e reativa adotada, aliada à infraestrutura em nuvem, viabiliza futuras extensões, como integração com APIs de supermercados, implementação de sincronização \textit{offline-first} e expansão para plataformas web e desktop. 

Em termos de contribuição prática, o sistema se posiciona como uma ferramenta de apoio à gestão doméstica e à educação financeira familiar, oferecendo uma interface simples e eficiente para o controle de gastos e a tomada de decisão. No campo acadêmico, o trabalho reforça o potencial do desenvolvimento multiplataforma com o uso de tecnologias emergentes como o \textit{Flutter} e o \textit{Firebase}, alinhando inovação tecnológica a um problema cotidiano de relevância social.

Este trabalho está organizado da seguinte forma: o Capítulo~\ref{cap:objetivos} apresenta os objetivos; o Capítulo~\ref{cap:fundamentacao} traz a fundamentação teórica, destacando os conceitos de computação em nuvem, desenvolvimento multiplataforma e colaboração digital; o Capítulo~\ref{cap:metodologia} descreve a metodologia adotada; o Capítulo~\ref{cap:desenvolvimento} detalha o processo de implementação; o Capítulo~\ref{cap:resultados} apresenta os resultados obtidos e as discussões; e, por fim, o Capítulo~\ref{cap:conclusao} sintetiza as conclusões e contribuições do trabalho.

Conforme reforça \textcite{pressman2016}, o desenvolvimento de sistemas modernos requer integração contínua entre projeto, implementação e validação — princípio que fundamentou cada etapa deste trabalho, desde a concepção até os testes finais.
\vspace{12pt}

% 2 - Objetivos
% ----------------------------------------------------------
\chapter{Objetivos}\label{cap:objetivos}
% ----------------------------------------------------------

O desenvolvimento de soluções digitais voltadas à organização doméstica e ao consumo consciente exige não apenas recursos tecnológicos, mas também uma estrutura que favoreça a colaboração e o controle inteligente de informações. Dessa forma, este capítulo define o objetivo central e os objetivos específicos do presente trabalho, alinhando o escopo técnico à relevância prática e acadêmica da pesquisa.

\section{Objetivo Geral}

Desenvolver um aplicativo multiplataforma para gerenciamento inteligente de listas de compras, utilizando o \textit{framework Flutter} e os serviços do \textit{Firebase}, de forma a permitir controle dinâmico de itens, organização por categorias e sincronização em tempo real dos dados do usuário. O sistema visa otimizar o processo de planejamento de compras, reduzir desperdícios e facilitar a gestão individual das informações de consumo.

\section{Objetivos Específicos}

Para atingir o objetivo geral proposto, o trabalho busca cumprir as seguintes metas técnicas e funcionais:

\begin{itemize}
    \item \textbf{Projetar e implementar uma arquitetura modular e escalável}, estruturada em camadas (apresentação, controle e modelo), aplicando boas práticas de engenharia de software e promovendo manutenção facilitada e baixo acoplamento entre módulos;
    
    \item \textbf{Implementar autenticação e gerenciamento de usuários} utilizando o serviço \textit{Firebase Authentication}, assegurando privacidade e integridade dos dados por meio de mecanismos de autenticação baseados em credenciais seguras;

    \item \textbf{Modelar e estruturar os dados no \textit{Cloud Firestore}} para representar listas, itens e categorias, garantindo consistência, escalabilidade e atualização em tempo real;
    
    \item \textbf{Assegurar sincronização imediata dos dados do usuário} por meio de \textit{listeners} do Firestore, permitindo que modificações realizadas nas listas sejam refletidas rapidamente na interface;

    \item \textbf{Construir interface e fluxo de navegação} aplicando o \textit{framework} GetX para gerenciamento de estado, controle de rotas e reatividade entre as camadas de apresentação e lógica de negócio;

    \item \textbf{Documentar o processo de desenvolvimento e as decisões técnicas adotadas}, conforme as normas acadêmicas e metodológicas, garantindo reprodutibilidade e clareza no processo de construção do sistema;

    \item \textbf{Propor diretrizes para evoluções futuras}, incluindo notificações push, sincronização \textit{offline-first} e integração de novos recursos que ampliem a experiência do usuário.
\end{itemize}

O conjunto desses objetivos visa não apenas a entrega de um produto funcional, mas também a consolidação de uma base metodológica e tecnológica capaz de servir como referência para futuras aplicações colaborativas orientadas a dados e acessíveis em múltiplas plataformas.


% 3 - Fundamentação Teórica
% ----------------------------------------------------------
\chapter{Fundamentação Teórica}\label{cap:fundamentacao}
% ----------------------------------------------------------

O desenvolvimento de aplicações móveis modernas está diretamente relacionado à capacidade de criar soluções acessíveis, performáticas e compatíveis com múltiplas plataformas. Com o avanço das tecnologias de computação em nuvem e dos \textit{frameworks} multiplataforma, tornou-se possível desenvolver sistemas colaborativos em tempo real, com custos reduzidos e alta escalabilidade.  
Nesta seção são abordados os conceitos fundamentais que embasam o desenvolvimento do aplicativo proposto, envolvendo desenvolvimento multiplataforma, computação em nuvem, sincronização em tempo real e princípios de usabilidade.

\section{Desenvolvimento Multiplataforma e o Ecossistema Flutter}

O desenvolvimento multiplataforma tem como objetivo permitir que uma única base de código seja executada em diferentes sistemas operacionais, como Android, iOS, Web e Desktop, reduzindo tempo e custo de desenvolvimento. Essa abordagem é especialmente relevante em contextos onde a agilidade de entrega e a uniformidade da experiência do usuário são fatores críticos \parencite{sommerville2011}.  

O \textit{Flutter}, \textit{framework} de código aberto mantido pela Google, utiliza a linguagem \textit{Dart} e o motor gráfico \textit{Skia} para renderização direta da interface, dispensando a ponte com componentes nativos (\textit{bridge}) e, consequentemente, entregando desempenho próximo ao nativo \parencite{googleflutter, derivaz2020}.  
Entre seus diferenciais, destacam-se o recurso de \textit{hot reload}, que permite a atualização instantânea da interface sem reinicialização do aplicativo, e a forte aderência ao \textit{Material Design}, garantindo consistência visual e responsividade entre plataformas.

Segundo \textcite{pressman2016}, frameworks modernos como o Flutter possibilitam maior produtividade ao abstrair detalhes de baixo nível do sistema operacional, permitindo ao desenvolvedor concentrar-se na lógica de negócio e na experiência do usuário.  
Além disso, o ecossistema Flutter fornece uma ampla gama de bibliotecas oficiais e comunitárias, integrando-se facilmente a serviços externos como \textit{Firebase}, APIs REST e bancos de dados locais.

O uso dessa tecnologia, portanto, oferece uma combinação de desempenho, portabilidade e agilidade, elementos fundamentais para o desenvolvimento do aplicativo proposto.

\section{Firebase e Computação em Nuvem}

A computação em nuvem, segundo \textcite{armbrust2010}, é o paradigma tecnológico que permite o acesso a recursos computacionais sob demanda, de forma escalável e distribuída. Essa abordagem elimina a necessidade de infraestrutura local, promovendo elasticidade e alta disponibilidade.  

O \textit{Firebase}, plataforma da Google baseada em nuvem, oferece um conjunto integrado de serviços voltados para o desenvolvimento de aplicações modernas, incluindo autenticação de usuários, banco de dados em tempo real, armazenamento de arquivos, notificações e hospedagem \parencite{googlefirebase}.  
Seu banco de dados principal, o \textit{Cloud Firestore}, utiliza um modelo NoSQL orientado a documentos, o que proporciona flexibilidade estrutural e baixa latência para operações de leitura e escrita.

Além disso, o \textit{Firebase Authentication} simplifica a implementação de mecanismos de login seguros.  
Essas características tornam o \textit{Firebase} uma escolha estratégica para aplicações móveis colaborativas, reduzindo a complexidade do back-end e facilitando o desenvolvimento de funcionalidades em tempo real.

Segundo \textcite{sommerville2011}, a adoção de plataformas em nuvem é uma tendência consolidada na engenharia de software moderna, pois combina agilidade de desenvolvimento com escalabilidade operacional. No contexto deste trabalho, o uso do \textit{Firebase} contribui diretamente para a redução de custos, simplificação da arquitetura e viabilização de recursos colaborativos.

\section{Sincronização em Tempo Real e Colaboração Digital}

Aplicações colaborativas exigem sincronização imediata dos dados entre diferentes dispositivos e usuários. Esse comportamento é garantido por tecnologias que implementam o conceito de reatividade, onde o sistema responde instantaneamente a mudanças de estado.  
No caso do \textit{Firebase}, o \textit{Cloud Firestore} disponibiliza \textit{listeners} para documentos e coleções, notificando automaticamente os clientes sobre alterações em tempo real. Esse modelo de atualização contínua evita inconsistências e elimina a necessidade de requisições manuais de atualização.

De acordo com \textcite{pressman2016}, a colaboração digital efetiva depende da integração entre camadas de software e do feedback imediato entre módulos, o que promove transparência e confiança no uso do sistema.  
Em aplicações domésticas ou corporativas, esse tipo de arquitetura reativa favorece o trabalho conjunto e o compartilhamento de informações, reduzindo o esforço de coordenação e aumentando a eficiência.

A sincronização em tempo real também está diretamente ligada à experiência do usuário, pois influencia a sensação de fluidez e a percepção de confiabilidade da aplicação.  
Com a combinação de eventos assíncronos, autenticação segura e controle de permissões, torna-se possível oferecer um ambiente colaborativo robusto, adequado para o gerenciamento conjunto de listas de compras, como proposto neste trabalho.

\section{Usabilidade, Experiência do Usuário e Análise de Consumo}

A usabilidade é um dos pilares da qualidade de software e está diretamente associada à facilidade com que o usuário realiza suas tarefas no sistema.  
Segundo \textcite{nielsen1994}, um produto é considerado usável quando possibilita que usuários específicos alcancem objetivos determinados com eficácia, eficiência e satisfação em um contexto de uso definido.  

Princípios de design centrado no usuário, como visibilidade do estado do sistema, consistência, prevenção de erros e feedback imediato, foram aplicados na concepção da interface do aplicativo.  
O uso das diretrizes do \textit{Material Design} contribuiu para uniformidade visual, hierarquia de informações clara e responsividade em diferentes dispositivos.

Além disso, a integração de análises de consumo dentro da aplicação promove uma camada de valor adicional, pois permite ao usuário compreender seus padrões de compra e identificar oportunidades de economia.  
Essas análises, quando associadas a visualizações gráficas interativas, ajudam na formação de hábitos de consumo mais conscientes e na organização financeira familiar, alinhando-se à proposta social do sistema desenvolvido.

\section{Considerações Finais da Fundamentação}

A literatura e os conceitos apresentados sustentam as decisões técnicas e arquiteturais tomadas no desenvolvimento deste projeto.  
A combinação entre desenvolvimento multiplataforma, computação em nuvem e sincronização em tempo real compõe um ecossistema eficiente, escalável e de rápida entrega.  
Ao integrar usabilidade e análise de dados, o sistema transcende o papel de simples aplicativo de lista de compras, posicionando-se como uma ferramenta de apoio à gestão colaborativa e ao consumo consciente.

% 4 - Metodologia
% ----------------------------------------------------------
\chapter{Metodologia}\label{cap:metodologia}
% ----------------------------------------------------------

A metodologia adotada para o desenvolvimento deste trabalho foi orientada por princípios ágeis, priorizando entregas incrementais e a validação contínua das funcionalidades implementadas. Essa abordagem buscou conciliar flexibilidade e controle, garantindo evolução iterativa do produto com foco em valor entregue ao usuário final.

O processo contemplou as etapas de planejamento, levantamento de requisitos, modelagem, implementação, testes e análise crítica. A natureza iterativa dessa metodologia possibilitou o refinamento progressivo da aplicação, com revisões contínuas de código, arquitetura e experiência de uso.

\section{Abordagem de Desenvolvimento}

A condução do projeto seguiu práticas inspiradas no framework \textit{Scrum}, amplamente utilizado em projetos de software que demandam adaptação rápida e foco em resultados tangíveis. Foram definidos ciclos curtos de desenvolvimento (\textit{sprints}) com duração média de uma a duas semanas, cada um abrangendo planejamento, execução e retrospectiva.

As principais práticas adotadas incluíram:
\begin{itemize}
\item \textbf{Planejamento de Sprints:} definição de tarefas prioritárias a partir do \textit{product backlog};
\item \textbf{Revisões Periódicas:} análise dos incrementos entregues e ajustes de escopo conforme a viabilidade técnica e os testes de usabilidade;
\item \textbf{Integração Contínua:} versionamento constante de código no \textit{GitHub}, garantindo rastreabilidade e reversão rápida em caso de falhas;
\item \textbf{Validação Contínua:} execução de testes funcionais e revisão de interface a cada ciclo de entrega.
\end{itemize}

Essa abordagem reforçou os princípios de adaptabilidade e comunicação contínua, aspectos centrais das metodologias ágeis \parencite{pressman2016, sommerville2011}. A divisão do trabalho em incrementos curtos permitiu verificar rapidamente o impacto das decisões técnicas e de design, minimizando retrabalho e fortalecendo a coesão entre as camadas da aplicação.

\section{Ferramentas e Tecnologias}

O desenvolvimento utilizou tecnologias e frameworks que suportam aplicações móveis modernas, de alta responsividade e integradas a serviços em nuvem. A Tabela~\ref{tab:ferramentas} sintetiza as principais ferramentas empregadas e suas respectivas funções no projeto.

\begin{table}[htbp]
\centering
\caption{Ferramentas e tecnologias utilizadas}
\label{tab:ferramentas}
\begin{tabular}{p{4cm} p{9cm}}
\toprule
\textbf{Ferramenta/Tecnologia} & \textbf{Função Principal} \\ 
\midrule
\textit{Flutter}/\textit{Dart} & Desenvolvimento do aplicativo móvel multiplataforma e interface responsiva. \\ 
\textit{Firebase Authentication} & Autenticação de usuários e gerenciamento seguro de sessões. \\ 
\textit{Cloud Firestore} & Armazenamento de dados em tempo real, estruturado em coleções e documentos. \\ 
\textit{Git/GitHub} & Controle de versão, integração contínua e histórico de desenvolvimento. \\ 
\bottomrule
\end{tabular}
\end{table}

A escolha desse conjunto de ferramentas foi motivada pela integração nativa entre os serviços Firebase e o framework Flutter, reduzindo a complexidade da configuração e aumentando a eficiência no desenvolvimento e na manutenção.

\section{Arquitetura Lógica}

A arquitetura do sistema foi estruturada de forma modular, adotando o padrão MVC (\textit{Model--View--Controller}) combinado com princípios reativos providos pelo \textit{GetX}. Essa escolha visou garantir desacoplamento entre camadas e escalabilidade no crescimento da base de código.

As camadas principais foram definidas da seguinte forma:
\begin{itemize}
\item \textbf{Apresentação (View):} telas e componentes visuais implementados em \textit{Flutter}, responsáveis pela interação com o usuário e pela exibição de dados;
\item \textbf{Controle (Controller):} gerenciamento do estado da aplicação e coordenação entre interface e lógica de negócio, utilizando o pacote \textit{GetX} para reatividade;
\item \textbf{Modelo (Model):} representação das entidades de domínio e mapeamento de dados persistidos no \textit{Firestore}.
\end{itemize}

Essa organização possibilita manutenção independente das camadas, além de favorecer a aplicação de testes unitários e a reutilização de componentes. A reatividade do Firestore garante atualização automática da interface sempre que há alteração nos dados remotos.

\section{Etapas Executadas}

O desenvolvimento foi dividido em etapas bem definidas, interdependentes e documentadas para garantir rastreabilidade das decisões técnicas:

\begin{enumerate}
\item \textbf{Planejamento e Levantamento de Requisitos:} identificação das funcionalidades essenciais e das restrições técnicas;
\item \textbf{Prototipação:} construção de interfaces no \textit{Figma} e definição dos fluxos de navegação principais;
\item \textbf{Implementação Incremental:} codificação modular, com commits versionados e entregas parciais validadas a cada sprint;
\item \textbf{Integração com Serviços Firebase:} autenticação, armazenamento em nuvem;
\item \textbf{Testes Funcionais e de Usabilidade:} execução em dispositivos físicos e emuladores, com foco na estabilidade e responsividade;
\item \textbf{Ajustes e Otimizações:} refinamento do desempenho, correção de inconsistências e melhorias de interface;
\item \textbf{Validação Final:} revisão da experiência do usuário e adequação dos critérios de qualidade definidos.
\end{enumerate}

Cada uma dessas fases foi iterada com revisões constantes, permitindo adaptar o produto à medida que o entendimento das necessidades evoluía.

\section{Critérios de Avaliação}

A validação dos resultados considerou métricas objetivas e subjetivas, de modo a avaliar o desempenho técnico e a experiência do usuário de forma equilibrada.

\begin{itemize}
\item \textbf{Desempenho:} medição de latência percebida e tempo de resposta nas operações de leitura e escrita no \textit{Firestore};
\item \textbf{Estabilidade:} observação de falhas, travamentos e comportamento sob múltiplas conexões simultâneas;
\item \textbf{Usabilidade:} análise da clareza dos fluxos, consistência visual e esforço de navegação;
\item \textbf{Satisfação:} coleta de feedback qualitativo de usuários sobre a facilidade de uso e utilidade percebida;
    \item \textbf{Segurança:} verificação das regras de acesso e conformidade com práticas recomendadas de autenticação.
\end{itemize}

Esses critérios garantiram uma visão abrangente da eficácia do sistema tanto do ponto de vista técnico quanto da experiência do usuário, consolidando a qualidade do produto final.

\subsection{Estratégia de Testes e Validação}

A validação do aplicativo foi realizada por meio de uma abordagem pragmática focada em testes manuais e de usabilidade, garantindo que o produto final atendesse aos requisitos funcionais e de experiência do usuário. Dada a natureza do projeto como um Protótipo de Mínima Viabilidade (MVP), a prioridade foi a validação em cenários de uso real.

A estratégia foi dividida em duas frentes principais:

\begin{itemize}
    \item \textbf{Testes Funcionais Manuais:} Para cada ciclo de desenvolvimento (\textit{sprint}), foi executada uma série de testes manuais para verificar o comportamento das funcionalidades implementadas. Os cenários de teste críticos incluíam:
    \begin{itemize}
        \item Fluxo completo de autenticação: cadastro, login, logout e recuperação de senha.
        \item Ciclo de vida das listas: criação, adição/remoção de itens, marcação de itens como comprados e finalização da lista.
        \item Colaboração em tempo real: compartilhamento de uma lista com outro usuário e verificação da sincronização de dados (adição de item por um usuário refletindo na tela do outro).
        \item Geração de análises: verificação se os gráficos de gastos eram gerados corretamente após a finalização de uma lista.
    \end{itemize}
    Esses testes foram realizados em múltiplos ambientes, incluindo emuladores do Android Studio e dispositivos físicos (Android), para garantir a consistência do comportamento.

    \item \textbf{Testes de Usabilidade:} Após a estabilização das funcionalidades principais, foram conduzidas sessões de teste de usabilidade com um grupo de 5 usuários-alvo, número que, segundo \textcite{nielsen2000}, é suficiente para revelar a maioria dos problemas de usabilidade de uma interface. O objetivo era avaliar a intuitividade da interface e a facilidade de aprendizado. Durante as sessões, os usuários foram convidados a realizar um conjunto de tarefas (como criar uma lista, convidar um amigo e finalizar uma compra) sem receber instruções detalhadas. O feedback qualitativo, como comentários e dificuldades observadas, foi coletado e utilizado para refinar a interface e o fluxo de navegação antes da versão final.
\end{itemize}

A ausência de testes automatizados (unitários, de widget e de integração) é reconhecida como uma limitação técnica, típica de fases iniciais de desenvolvimento. No entanto, a abordagem de testes manuais e de usabilidade se mostrou eficaz para garantir a qualidade, a estabilidade e a aderência do MVP aos objetivos do projeto.

% 5 - Desenvolvimento
% ----------------------------------------------------------
\chapter{Desenvolvimento}\label{cap:desenvolvimento}
% ----------------------------------------------------------

Este capítulo descreve o processo de implementação do aplicativo desenvolvido em \textit{Flutter}, com integração direta ao ecossistema \textit{Firebase}. O objetivo é detalhar as decisões técnicas, a estrutura do projeto e as principais funcionalidades que materializam a proposta de gerenciamento inteligente e colaborativo de listas de compras.

O desenvolvimento seguiu uma abordagem incremental e iterativa, fundamentada nos princípios de engenharia de software moderna \parencite{pressman2016}, e na adoção de boas práticas arquiteturais, conforme recomendam \textcite{sommerville2011} e \textcite{bass2012}.

\section{Apoio ao Desenvolvimento}

\subsection{Estrutura e Arquitetura do Projeto}

A arquitetura adotada é uma adaptação da Arquitetura Limpa (\textit{Clean Architecture}) \parencite{martin2017}, com forte ênfase na modularização por funcionalidade (\textit{feature}). O objetivo principal é separar o código em camadas com responsabilidades distintas. A regra mais importante é que as dependências apontam sempre para ``dentro'', ou seja, camadas externas conhecem as internas, mas as internas não conhecem as externas.

\subsubsection{Camada de Apresentação (Presentation)}

\begin{itemize}
    \item \textbf{Localização:} Principalmente na pasta \texttt{app/features/}.
    \item \textbf{Papel:} Parte visual e interativa da aplicação com a qual o usuário interage.
    \item \textbf{Organização:} O projeto é dividido em módulos baseados em funcionalidades. Cada subpasta dentro de \texttt{features} (como \texttt{auth}, \texttt{home}, \texttt{product}, \texttt{shopping\_list}) é um módulo autônomo, contendo:
    \begin{itemize}
        \item \textbf{Views/Pages:} Telas que o usuário visualiza.
        \item \textbf{Controllers:} Lógica que controla o estado da tela, responde a interações do usuário e busca os dados necessários para exibição.
    \end{itemize}
\end{itemize}

\subsubsection{Camada de Domínio (Domain)}

\begin{itemize}
    \item \textbf{Localização:} Mais conceitual; regras de negócio e orquestração ficam nos Controllers (dentro de \texttt{features}), e os contratos são definidos pelas abstrações dos Repositories.
    \item \textbf{Papel:} Contém as regras de negócio puras e orquestra o fluxo de dados para a interface do usuário. O Controller de uma feature solicita os dados via interface de um Repository, sem conhecer a implementação concreta.
    \item \textbf{Ponto-chave:} O Controller não sabe se os dados vêm do \textit{Firebase}, de uma API REST ou de um banco local; ele apenas solicita e confia no contrato do Repository.
\end{itemize}

\subsubsection{Camada de Dados (Data)}

\begin{itemize}
    \item \textbf{Localização:} Pasta \texttt{app/data/}.
    \item \textbf{Papel:} Responsável por buscar, salvar e gerenciar dados, de fontes remotas ou locais.
    \item \textbf{Organização:}
    \begin{itemize}
        \item \texttt{models}: Define os objetos e estruturas de dados (ex.: Produto com id, nome e preço).
        \item \texttt{repositories}: Implementação concreta dos contratos. Contém a lógica de decidir de onde buscar os dados (cache ou fonte remota) usando os providers.
        \item \texttt{providers}: Parte mais externa da aplicação, responsável pela comunicação direta com o mundo exterior (ex.: chamadas ao \textit{Firestore}, API do \textit{Firebase Auth}, etc.).
    \end{itemize}
\end{itemize}

\subsubsection{Resumo do Fluxo e Organização}

\begin{enumerate}
    \item \textbf{Módulos por Feature:} A organização principal é por funcionalidade (\texttt{auth}, \texttt{home}, etc.), facilitando encontrar tudo relacionado a uma parte específica do aplicativo.
    \item \textbf{Separação por Camadas:} Dentro dessa organização, aplicam-se os princípios da Arquitetura Limpa.
    \item \textbf{Fluxo de Dependência:} Sempre \texttt{Controller (Feature)} $\rightarrow$ \texttt{Repository (Contrato)} $\rightarrow$ \texttt{Repository (Implementação)} $\rightarrow$ \texttt{Provider}. A interface do usuário depende da lógica de dados, mas a lógica de dados não depende da interface.
\end{enumerate}

\section{Modelagem de Dados}

A modelagem segue um padrão NoSQL, otimizado para o Firestore, e é estruturada em coleções principais e sub-coleções, seguindo práticas recomendadas para bancos de dados não relacionais que priorizam a velocidade de leitura e a escalabilidade em detrimento da normalização de dados \parencite{sadalage2012}.

\subsection{Coleção \texttt{users}}
Esta coleção armazena o perfil público de cada usuário. O ID de cada documento é o mesmo UID fornecido pelo Firebase Authentication.

\begin{itemize}
    \item \texttt{id} (string): ID do usuário (o mesmo do Auth).
    \item \texttt{name} (string): Nome do usuário.
    \item \texttt{email} (string): Email do usuário.
    \item \texttt{phone} (string, opcional): Telefone do usuário.
    \item \texttt{photoUrl} (string, opcional): URL da foto de perfil.
\end{itemize}

\subsection{Coleção \texttt{categories}}
Armazena as categorias que os usuários criam para organizar suas listas de compras.

\begin{itemize}
    \item \texttt{name} (string): Nome da categoria (ex: "Supermercado Mensal", "Churrasco").
    \item \texttt{createdBy} (string): O ID do usuário que criou a categoria.
    \item \texttt{createdAt}, \texttt{updatedAt} (timestamp): Datas de criação e atualização.
\end{itemize}

\subsection{Coleção \texttt{products}}
Funciona como um catálogo de produtos pessoal para cada usuário.

\begin{itemize}
    \item \texttt{name} (string): Nome do produto (ex: "Arroz 5kg").
    \item \texttt{description} (string): Descrição do produto.
    \item \texttt{imageUrl} (string, opcional): URL da imagem do produto.
    \item \texttt{ownerId} (string): O ID do usuário que cadastrou o produto.
    \item \texttt{createdAt}, \texttt{updatedAt} (timestamp): Datas de criação e atualização.
\end{itemize}

\subsection{Coleção \texttt{shopping\_lists} (ou \texttt{lists})}
A coleção central do aplicativo, contendo as listas de compras.

\begin{itemize}
    \item \texttt{name} (string): Nome da lista (ex: "Compras de Janeiro").
    \item \texttt{ownerId} (string): O ID do usuário dono da lista.
    \item \texttt{status} (string): O estado atual da lista (ex: "ativa", "finalizada").
    \item \texttt{categoryId} (string): O ID da categoria à qual a lista pertence.
    \item \texttt{memberUIDs} (array de strings): Lista com os IDs de todos os usuários que são membros desta lista.
    \item \texttt{memberPermissions} (map): Mapeia o ID de um membro a sua permissão (ex: \{ "user\_id\_1": "editor" \}).
    \item \texttt{totalPrice} (double): Valor total da lista, calculado.
    \item \texttt{createdAt}, \texttt{purchaseDate} (timestamp): Datas de criação e da finalização da compra.
\end{itemize}

\subsection{Sub-coleção: \texttt{items}}
Cada lista de compras possui uma sub-coleção chamada \texttt{items} em vez de armazenar os produtos dentro de um array no documento da lista.  
A estrutura é: \texttt{shopping\_lists/\{id\_da\_lista\}/items/\{id\_do\_item\}}.

Um documento em \texttt{items} representa um produto na lista e possui os seguintes campos:

\begin{itemize}
    \item \texttt{productId} (string): O ID do produto original da coleção \texttt{products}.
    \item \texttt{productName}, \texttt{productImageUrl} (string): Dados duplicados (desnormalizados) do produto para acesso rápido sem outra consulta.
    \item \texttt{quantity} (double): Quantidade do item na lista.
    \item \texttt{unitPrice} (double): Preço unitário no momento da adição.
    \item \texttt{totalItemPrice} (double): Preço total do item (\texttt{quantity * unitPrice}).
    \item \texttt{isCompleted} (boolean): Indica se o item já foi "pego" no carrinho.
\end{itemize}

Essa abordagem com sub-coleção é mais escalável do que usar um array, permitindo que cada lista tenha um número virtualmente ilimitado de itens.

\section{Funcionalidades Chave}

\subsection{Autenticação e Gestão de Perfil}
\begin{itemize}
    \item \textbf{Cadastro de Conta:} Permite que novos usuários se cadastrem usando e-mail e senha.
    \item \textbf{Login e Logout:} Acesso seguro à conta e capacidade de sair dela.
    \item \textbf{Recuperação de Senha:} Funcionalidade de "esqueci minha senha" para redefini-la.
    \item \textbf{Gerenciamento de Perfil:} O usuário pode visualizar e, potencialmente, editar suas informações, como nome e foto.
\end{itemize}

\subsection{Tela Principal (Home Page)}
\begin{itemize}
    \item \textbf{Visão Geral das Listas:} A tela principal serve como o painel central do usuário, exibindo um resumo de suas listas de compras ativas.
    \item \textbf{Ponto de Partida:} É o ponto de partida para a maioria das interações, permitindo que o usuário navegue para uma lista existente ou inicie a criação de uma nova.
    \item \textbf{Feedback Imediato:} Mostra o estado atual das listas, como o nome e o progresso geral, facilitando o acesso rápido às compras em andamento.
\end{itemize}

\subsection{Gestão de Listas de Compras}
\begin{itemize}
    \item \textbf{Criação de Listas:} O usuário pode criar múltiplas listas de compras, atribuindo um nome e uma categoria a elas.
    \item \textbf{Adição de Itens:} É possível adicionar produtos do catálogo pessoal à lista, especificando a quantidade e o preço.
    \item \textbf{Interação com a Lista:} Dentro de uma lista ativa, o usuário pode marcar/desmarcar itens como ``comprados'', e o sistema calcula o preço total em tempo real.
    \item \textbf{Ciclo de Vida da Lista:} As listas possuem status (ex: ``ativa'', ``finalizada''), permitindo que sejam movidas para um histórico após a conclusão.
\end{itemize}

\subsection{Colaboração em Tempo Real}
\begin{itemize}
    \item \textbf{Compartilhamento de Listas:} O dono de uma lista pode convidar outros usuários para participarem dela.
    \item \textbf{Gerenciamento de Membros:} O dono pode definir permissões para os membros (ex: apenas visualizar ou editar a lista).
    \item \textbf{Sincronização Automática:} Todas as alterações feitas em uma lista compartilhada (como adicionar ou marcar um item) são refletidas em tempo real para todos os membros.
\end{itemize}

\subsection{Catálogo de Produtos Personalizado}
\begin{itemize}
    \item \textbf{Cadastro de Produtos:} Os usuários podem criar seu próprio catálogo de produtos, incluindo nome, descrição e uma foto.
    \item \textbf{Reutilização de Produtos:} Facilita a adição de itens recorrentes às listas sem precisar digitar tudo novamente.
    \item \textbf{Gerenciamento do Catálogo:} Permite editar ou remover produtos já cadastrados.
\end{itemize}

\subsection{Histórico e Análise de Gastos}
\begin{itemize}
    \item \textbf{Histórico de Compras:} O aplicativo mantém um registro de todas as listas que foram finalizadas.
    \item \textbf{Análise de Despesas:} Oferece visualizações por meio de gráficos interativos que permitem ao usuário analisar seus gastos ao longo do tempo, filtrando por período ou categoria.
\end{itemize}

\section{Usabilidade e Navegação}

\subsection{Tecnologia de Navegação: GetX}

O projeto utiliza o pacote \textbf{GetX} para gerenciamento de estado e, principalmente, para a navegação. Essa escolha impacta diretamente a usabilidade e a manutenção do aplicativo:

\begin{itemize}
    \item \textbf{Navegação por Rotas Nomeadas:} Em vez de chamar uma tela diretamente, navegamos por nomes de rotas (ex.: \texttt{/home}, \texttt{/login}, \texttt{/lista/123}). Isso desacopla a lógica de navegação das telas, seguindo boas práticas de arquitetura.
    \item \textbf{Centralização das Rotas:} Todas as rotas e suas transições estão definidas no arquivo: \texttt{/app} \texttt{/routes} \texttt{/app-pages.dart}, facilitando a manutenção e a visualização do fluxo do aplicativo.
    \item \textbf{Sintaxe Simplificada:} GetX permite navegar sem a necessidade de \textbf{BuildContext}, usando comandos como \textbf{Get.toNamed('/detalhes')} para avançar e \textbf{Get.back()} para retornar.
\end{itemize}

\subsection{Padrões de Navegação do Usuário}

A experiência do usuário é construída com base em padrões de navegação bem estabelecidos:

\begin{enumerate}
    \item \textbf{Menu de Navegação Principal (Drawer):} O aplicativo utiliza um menu lateral como hub principal, acessível de qualquer tela principal. Permite navegar para áreas-chave como:
    \begin{itemize}
        \item Home (tela inicial de listas)
        \item Produtos (catálogo)
        \item Categorias
        \item Histórico de Compras
        \item Análise de Gastos
        \item Perfil
    \end{itemize}

    \item \textbf{Navegação em Pilha (Stack Navigation):} Padrão comum para fluxos de tarefas, em que cada nova tela é "empilhada" sobre a anterior.
    \begin{itemize}
        \item Exemplo: Usuário na tela Home $\rightarrow$ toca em uma lista específica $\rightarrow$ a tela de Detalhes da Lista é empilhada $\rightarrow$ ao voltar, retorna para Home.
        \item Esse padrão cria um fluxo lógico e previsível, garantindo que o botão "voltar" do dispositivo funcione conforme esperado.
    \end{itemize}

    \item \textbf{Navegação por Abas (TabBar):} Em certas telas, como Home ou Histórico, podem ser utilizadas abas para alternar entre visualizações relacionadas.
    \begin{itemize}
        \item Exemplo: Na tela de listas, abas para ``Listas Ativas'' e ``Listas Finalizadas'' permitem alternar rapidamente sem sair da tela principal.
    \end{itemize}
\end{enumerate}

\subsection{Resumo da Usabilidade de Navegação}

A combinação de \textbf{Drawer} para navegação global, \textbf{Stack Navigation} para tarefas específicas e \textbf{abas} para conteúdo secundário cria uma arquitetura de informação clara e uma experiência intuitiva. O uso de \textbf{GetX} garante que essa navegação seja implementada de forma robusta e de fácil manutenção.

\subsection{Gerenciamento de Erros e Segurança}

A robustez do aplicativo é garantida por uma estratégia de segurança multicamada e um sistema de tratamento de erros projetado para fornecer feedback claro ao usuário.

\subsubsection{Tratamento de Erros}

O gerenciamento de erros é implementado principalmente na camada de \textbf{Controller}, que atua como intermediário entre a interface do usuário e a camada de dados. A abordagem adotada é a seguinte:

\begin{itemize}
    \item \textbf{Captura de Exceções Específicas:} Em operações críticas, como login e cadastro, o código utiliza blocos \texttt{try...catch} para capturar exceções específicas da plataforma Firebase, como \texttt{FirebaseAuthException}. Isso permite identificar erros comuns (ex: "e-mail já em uso", "senha inválida", "usuário não encontrado") e exibir mensagens personalizadas e informativas.

    \item \textbf{Feedback ao Usuário:} Os erros são comunicados ao usuário de forma não intrusiva através de \texttt{Get.snackbar} (do pacote GetX). Essa abordagem fornece feedback imediato sem interromper o fluxo de navegação, informando o usuário sobre o que deu errado (por exemplo, "Erro no Login: Senha inválida.").

    \item \textbf{Falha Silenciosa na Camada de Dados:} Na camada de repositório, como no \texttt{UserRepository}, as operações de leitura que falham (por exemplo, ao buscar um usuário que não existe ou devido a um problema de rede) são projetadas para retornar \texttt{null} em vez de propagar uma exceção. O Controller que chamou a função é então responsável por tratar o caso nulo, aumentando a resiliência do sistema a falhas de leitura.
\end{itemize}

\subsubsection{Estratégia de Segurança}

A segurança dos dados do usuário é uma prioridade e é garantida por meio de práticas recomendadas para aplicações baseadas em nuvem:

\begin{itemize}
    \item \textbf{Autenticação Delegada:} O aplicativo utiliza o \textbf{Firebase Authentication} para todo o gerenciamento de ciclo de vida de autenticação. Isso significa que nenhuma senha é armazenada ou manipulada diretamente pelo aplicativo. O Firebase gerencia de forma segura o armazenamento de credenciais (usando hashing e salting) e a emissão de tokens de sessão, seguindo os padrões de segurança da indústria.

    \item \textbf{Regras de Segurança do Firestore (Server-Side):} A principal linha de defesa do aplicativo está nas regras de segurança definidas no arquivo \texttt{firestore.rules}. Essas regras são aplicadas diretamente nos servidores do Firebase e garantem que as operações de leitura e escrita nos dados só sejam permitidas para usuários autenticados e autorizados. Por exemplo, as regras garantem que um usuário só possa modificar suas próprias listas de compras e que apenas o proprietário de uma lista possa convidar ou remover membros. Isso impede que um usuário mal-intencionado acesse ou modifique dados de outros usuários, mesmo que o código do aplicativo seja comprometido.

    \item \textbf{Validação de Entrada:} A validação de dados de entrada, como o formato do e-mail e a força da senha, é atualmente delegada ao Firebase. O aplicativo captura e exibe os erros de validação retornados pelo serviço de autenticação, informando ao usuário sobre a necessidade de correção dos dados.

% 6 - Resultados e Discussão
% ----------------------------------------------------------
\chapter{Resultados e Discussões}\label{cap:resultados}
% ----------------------------------------------------------

Os resultados obtidos durante a fase de testes validaram a proposta do sistema e demonstraram que o aplicativo atendeu de forma satisfatória aos objetivos funcionais e não funcionais definidos no planejamento. A aplicação manteve latências baixas, estabilidade sob uso simultâneo e apresentou comportamento previsível em cenários reais de colaboração. As análises qualitativas e quantitativas evidenciam a consistência técnica e a coerência das decisões de arquitetura adotadas.

\section{Síntese dos Resultados Funcionais}
O aplicativo suportou autenticação de usuários via \textit{Firebase Authentication}, criação e compartilhamento de listas entre múltiplos participantes, sincronização imediata de modificações e controle de permissões por nível de acesso. Também foi possível realizar o acompanhamento de gastos agregados, categorizados por tipo de item e período, oferecendo indicadores úteis para análise de consumo.  
Essas funcionalidades confirmam o cumprimento dos requisitos levantados na fase de análise, evidenciando maturidade do produto enquanto \textit{MVP} funcional e usável.

\section{Desempenho e Estabilidade}
Durante os testes de desempenho, o sistema foi submetido a cenários com múltiplos usuários editando simultaneamente uma mesma lista. As atualizações foram refletidas quase instantaneamente entre os dispositivos, com tempo médio de propagação inferior a 300 milissegundos. Não foram observados conflitos de escrita, demonstrando a eficácia do modelo reativo do \textit{Cloud Firestore} e das regras de segurança aplicadas.  
A arquitetura baseada em \textit{streams} e \textit{listeners} manteve a responsividade da interface, mesmo sob condições de rede instáveis, preservando a consistência dos dados. O consumo de recursos foi considerado adequado para o porte do aplicativo, com impacto energético moderado, típico de aplicações em tempo real.

\section{Usabilidade e Aderência}
Os testes de usabilidade indicaram boa aceitação por parte dos usuários participantes. As principais tarefas — criação de listas, adição de itens, marcação de compras e exclusão de registros — foram realizadas com baixo esforço cognitivo e número reduzido de interações.  
A interface, construída com base nas diretrizes do \textit{Material Design}, proporcionou experiência fluida e consistente entre plataformas Android e Web. A curva de aprendizado foi curta, e o sistema mostrou-se intuitivo mesmo para usuários com pouca familiaridade tecnológica. Esse fator reforça o potencial de adoção do aplicativo em ambientes domésticos e colaborativos.

\section{Análise Crítica}
A combinação entre \textit{Flutter} e \textit{Firebase} demonstrou ser eficiente para o desenvolvimento ágil de um produto multiplataforma, reduzindo o \textit{time-to-market} e facilitando a manutenção. A integração nativa entre autenticação, banco de dados e hospedagem simplificou a infraestrutura e minimizou o esforço de configuração.  
Entretanto, foram observadas limitações inerentes à arquitetura proposta: dependência de conectividade contínua para garantir a sincronização de dados, ausência de suporte nativo a operação \textit{offline-first} e leve aumento no consumo de energia em dispositivos móveis devido à manutenção constante de \textit{listeners} ativos.  
Essas restrições não comprometem a viabilidade do sistema, mas apontam oportunidades claras de otimização e evolução tecnológica.

\section{Implicações Práticas}
Os resultados obtidos reforçam o impacto prático da solução, tanto em aspectos técnicos quanto sociais. O sistema demonstrou potencial para otimizar a organização doméstica, promover o controle financeiro compartilhado e estimular o consumo consciente, ao oferecer transparência sobre hábitos de compra e padrões de gasto.  
Além disso, o projeto evidencia como tecnologias acessíveis e de rápido desenvolvimento podem ser aplicadas a problemas cotidianos com impacto real. A integração entre mobilidade, colaboração e nuvem representa uma tendência consolidada no desenvolvimento de soluções orientadas à praticidade e à conectividade constante.

Em síntese, os testes confirmaram a consistência da proposta e a maturidade técnica do protótipo desenvolvido. O desempenho obtido, aliado à boa experiência de uso, comprova a viabilidade do modelo e consolida a base para avanços futuros, tanto em funcionalidades quanto em abrangência de público e integração tecnológica.


% 7 - Conclusão e Contribuições
% ----------------------------------------------------------
\chapter{Conclusão e Contribuições do Trabalho}\label{cap:conclusao}
% ----------------------------------------------------------

O desenvolvimento deste projeto comprovou a viabilidade técnica e prática de um aplicativo multiplataforma, colaborativo e escalável para gestão de listas de compras, utilizando o \textit{framework Flutter} e os serviços em nuvem do \textit{Firebase}. A solução atendeu aos objetivos propostos, demonstrando eficiência na sincronização de dados, boa experiência de uso e potencial de expansão funcional. O sistema validou o conceito de um ambiente unificado para controle e colaboração em compras domésticas, entregando valor tanto técnico quanto social.

\section{Contribuições Técnicas}
\begin{itemize}
    \item Implementação de uma arquitetura modular baseada em estado reativo, integrando autenticação, persistência de dados e mensageria em tempo real;
    \item Uso de \textit{listeners} do Firestore para sincronização imediata entre dispositivos, com regras de segurança e controle de acesso definidas de forma consistente;
    \item Adoção de boas práticas de usabilidade e \textit{Material Design}, resultando em uma interface intuitiva e responsiva;
    \item Criação de uma base escalável para evolução futura do sistema, permitindo novas integrações e funcionalidades.
\end{itemize}

\section{Contribuições Sociais e Aplicadas}
\begin{itemize}
    \item Facilitação da colaboração entre membros de um mesmo núcleo familiar, promovendo organização e divisão de responsabilidades nas compras;
    \item Apoio ao consumo consciente, fornecendo visibilidade de gastos, itens recorrentes e padrões de compra;
    \item Inclusão digital por meio do acesso multiplataforma, possibilitando o uso em diferentes dispositivos e contextos;
    \item Estímulo à adoção de soluções tecnológicas simples e acessíveis para gestão doméstica.
\end{itemize}

\section{Limitações}
O sistema ainda depende de conectividade constante para funcionamento, não implementando um modelo \textit{offline-first}, o que restringe o uso em ambientes sem internet. Além disso, há consumo de energia levemente superior em dispositivos móveis devido aos \textit{listeners} contínuos. Por fim, o aplicativo ainda não possui integração direta com catálogos de supermercados nem recursos avançados de análise de consumo.

\section{Trabalhos Futuros}
\begin{itemize}
    \item Implementar suporte a modo \textit{offline-first}, com sincronização posterior automática;
    \item Implementar mecanismos de recomendação baseados em histórico;
    \item Adicionar um painel web administrativo com relatórios e métricas de uso;
    \item Incorporar funcionalidades de acessibilidade, suporte multilíngue e notificações inteligentes.
\end{itemize}

\subsection{Considerações sobre Escalabilidade Futura}

A arquitetura de software adotada no projeto foi escolhida não apenas para a implementação dos requisitos atuais, mas também com a escalabilidade futura em mente. A estrutura baseada nos princípios da Arquitetura Limpa (\textit{Clean Architecture}) e a organização do código por funcionalidades (\textit{features}) são fatores-chave que facilitam a evolução e manutenção do sistema.

A estrita separação entre as camadas de Apresentação, Domínio e Dados permite que futuras alterações sejam feitas de forma isolada. Por exemplo, a implementação de um modo \textit{offline-first} (um dos trabalhos futuros propostos) pode ser realizada majoritariamente na camada de \textbf{Dados}, modificando os repositórios para que busquem dados de um cache local (como um banco de dados SQLite ou Isar) antes de consultar o Firestore, sem a necessidade de reescrever a lógica de negócio ou a interface do usuário.

Da mesma forma, a modelagem de dados no Firestore, que utiliza sub-coleções para os itens das listas, é inerentemente escalável, suportando um número virtualmente ilimitado de itens por lista sem degradação de desempenho ou risco de exceder os limites de tamanho de documento.

Essas decisões arquiteturais garantem que a adição de novas funcionalidades complexas, como sistemas de recomendação ou painéis administrativos, possa ser feita de maneira modular, minimizando o débito técnico e permitindo que o aplicativo cresça de forma sustentável para atender a uma base de usuários maior e a requisitos mais exigentes.

Em síntese, o trabalho apresenta um \textit{MVP} tecnicamente sólido, com valor social evidente e potencial real de escalabilidade. As soluções desenvolvidas demonstram aplicabilidade imediata e fornecem uma base robusta para continuidade em projetos de pesquisa, inovação ou produto comercial.


\postextual
\begingroup
\printbibliography[title=REFERÊNCIAS]
\endgroup

\end{document}
